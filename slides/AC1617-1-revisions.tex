\documentclass{beamer}

\input{macros/beamerconf}
\input{macros/preamble}
\input{macros/macros}

%----------------------------------------------------------------------------
\usepackage{graphicx,amsmath}
\usepackage{stmaryrd} % cf. interleave
%\usepackage{./macros/myisolatin1}

\usepackage{amscd}
%------ using xy ------------------------------------------------------------
\usepackage{alltt}
%------ using xy ------------------------------------------------------------
\usepackage[all]{xy}
%\def\larrow#1#2#3{\xymatrix{ #3 & #1 \ar[l] _-{#2} }}
\def\larrow#1#2#3{\xymatrix{ #3 & #1 \ar[l] _--{#2} }}
\def\rarrow#1#2#3{\xymatrix{ #1 \ar[r]^-{#2} & #3 }}
\def\arLaw#1#2#3#4#5{
\xymatrix{
        #1      \ar@/^1pc/[rr]^-{#4} &
        #5 &
        #2      \ar@/^1pc/[ll]^-{#3}
}}
\def\arLeq#1#2#3#4{\arLaw{#1}{#2}{#3}{#4}\leq}
%------ using pstricks (rnode etc) ------------------------------------------
%\usepackage{pstricks,pst-node,pst-text,pst-3d}



\def\laplace#1#2{*\txt{\mbox{ \fcolorbox{black}{myGray}{$\begin{array}{c}\mbox{#1}\\\\#2\\\\\end{array}$} }}}

\def\SING{\ensuremath{\boldsymbol{1}}}
\def\para{\mathbin{\boxtimes}}   % Mealy produto (sincrono)
\def\parc{\mathbin{\boxplus}}    % Mealy soma
\def\pars{\mathbin{\boxast}}     % Mealy misto
\def\shk#1#2{\bigl(#1\bigr)_{#2}}            % separated hook direita
\def\hk#1#2{#1\! \Lsh_{#2}}            % hook direita
\def\hkespaco#1#2{#1 \Lsh_{#2}}            % hook direita
%\def\feed#1{\Bigl( #1 \Bigr)\! \Lsh}            % hook direita
\def\feed#1{#1 \Lsh}            % hook direita
\def\kh#1#2{~_{#2}(#1)}           % hook esquerda

\def\flift#1{\ulcorner #1 \urcorner}           % funcoes embb
\def\singr#1{\lfloor #1 \rceil}

\def\pdiag{\bigtriangleup}      % diagonal em componentes
\def\pcodiag{\bigtriangledown}    % codiagonal em componentes

\def\qcomp{\mathbin{\boldsymbol{;}}}
\def\qcom#1#2{#1\qcomp#2}
\def\sync{}



\title{
	Software architecture for reactive systems: \\
  A short look at model checking (revisions)
	}
\author{Jos\'{e} Proen\c{c}a}
\institute{HASLab - INESC TEC \\ Universidade do Minho\\ Braga, Portugal}

\date{
\begin{tabular}{c}
\\
    February  2017
\end{tabular}
}


\begin{document}

\frame[plain]{\titlepage}

\section{Ideas}

%%%%%%%%%%%%%%%%%%%%%%%
\begin{frame}{Model checking}
  Recall ``Especifica��o e Modela��o'':
  \begin{itemize}
  \item \red{Modelling} reactive systems -- \blue{Kripke structures} and
      \begin{tabular}[t]{@{}l@{}}\blue{\sout{Petri Nets}}\\\blue{SMV}\end{tabular}
  \item \red{Specification} -- Temporal logics (\blue{LTL} \sout{and CTL/CTL$^{*}$})
  \item \red{Verification} -- Check if a formula holds in a system 
  \end{itemize}

  ~\\[5mm]
  \myblock{SMV model checker}
\end{frame}

%%%%%%%%%%%%%%%%%%%%%%%
\begin{frame}{What we will see}
  \begin{itemize}
    \item \blue{Labelled transition systems (LTS)} as Kripke structures\\
    \begin{itemize}
      \item \blue{Process algebra} (not \sout{Petri Nets} SMV) to define LTS
      \item \red{mCRL2} toolset to model (not SMV)
      \item Equivalence of LTS
    \end{itemize}
    \item \blue{Modal logics} -- generalising temporal logics (CTL$^{*}$,CTL,LTL)
    \item Using \red{mCRL2} toolset to \blue{verify} properties
  \end{itemize}
  ~\\[5mm]
  \begin{itemize}
    \item Later: \blue{Timed-automata} and \red{UPPAAL} model checker (CTL)\\
  \end{itemize}
\end{frame}

%%%%%%%%%%%%%%%%%%%%%%%
\begin{frame}{Model}
  \myblock{$\ger{M}, w ~\models~ \phi$  ~~--~~what does it mean? }
  \begin{block}{Model definition}
    A \dkb{model} for the language is a pair $\dkb{\ger{M}} = \pair{\ger{F},V}$, where
    \begin{itemize}
    \item $\ger{F} = \pair{\dgold{W}, \enset{\alert{R_m}}_{m \in \mathsf{MOD}}}$  \\ is a \dkb{Kripke frame}, ie, a non empty set \dgold{$W$} and a family \alert{$R_m$} of \alert{binary relations} (called \emph{accessibility relations})
    over \dgold{$W$}, one for each modality symbol $m \in \mathsf{MOD}$. Elements of \dgold{$W$} are called \dgold{points},  \dgold{states},  \dgold{worlds} or
    simply  \dgold{vertices} in  directed graphs. 
    \item $\fdec{V}{\mathsf{PROP}}{\pow{(W)}}$ is a \dkb{valuation}.
    \end{itemize}
  \end{block}

  \begin{exampleblock}{Kripke structures from last semester}
  \begin{itemize}
    \item MOD = $\set{\textbf{1}}$
    \item $(S, I, R, L)$ ~~where~~ $S=W$, $I=\set{w}$, $R=R_{\textbf{1}}$, $L =V$
    \item \dgold{$\ger{F} = \pair{W,R}$} ~~instead of~~
          \dkb{$\ger{F} = \pair{{W}, \enset{{R_m}}_{m \in \mathsf{MOD}}}$}
  \end{itemize}
  \end{exampleblock}
\end{frame}

%%%%%%%%%%%%%%%%
\begin{frame}{Example}
  \centering
  $\ger{M} = \wrap{\xymatrix{
  & 1^{p}  \ar[ld]_{\blue a}  \ar[rd]^{\blue a}\\
  2^{q} \ar[rr]^{\blue b}  && 3^{p,q} \ar@(ur,dr)[]^{\blue b}
  }}$
  \\[5mm]
  \begin{columns}
    \column{.45\textwidth}
    \begin{align*}
      W =& \set{1,2,3}\\
      MOD =& \set{\blue{a},\blue b}\\
      R_{\blue a} =& \set{(1,2),(1,3)}\\
      R_{\blue b} =& \set{(2,3),(3,3)}\\
      V =& \{1\mapsto \set{p},\\
         & ~\, 2\mapsto \set{q},\\
         & ~\, 3\mapsto \set{p,q}\}
    \end{align*}
    \column{.45\textwidth}
    \begin{itemize}
      \item \red{$\ger{M},1 ~\models~ p$}\\
      means $p$ holds in state 1
      \item \red{$\ger{M},2 ~\models~ [b]\,p$}\\
      means $p$ holds in every state reachable with $b$ from 2.
    \end{itemize}
  \end{columns}

\end{frame}


%%%%%%%%%%%%%%
\begin{frame}{Key differences}
  \begin{block}{Before}
    \begin{columns}
      \column{.40\textwidth}\centering
      $\wrap{\xymatrix{
        & 1^{\red p}  \ar[ld]  \ar[rd]\\
        2^{\red q} \ar[rr]  && 3^{\red{p,q}} \ar@(ur,dr)[]
        }}$
      \column{.56\textwidth}
      \begin{itemize}
        \item emphasize on \blue{states} - desired/forbidden states
        \item \blue{SMV language} to generate models
        \item $\ger{M},1 \,\models\, \red{p}$ ~,~ $\ger{M},1 \,\models\, \red{F\,G\,p}$
      \end{itemize}
    \end{columns}
  \end{block}
%
  \begin{block}{Now}
    \begin{columns}
      \column{.40\textwidth}\centering
      $\wrap{\xymatrix{
          & 1  \ar[ld]_{\red a}  \ar[rd]^{\red a}\\
          2 \ar[rr]^{\red b}  && 3 \ar@(ur,dr)[]^{\red b}
        }}$
      \column{.56\textwidth}
      \begin{itemize}
        \item emphasize on \blue{actions} - desired/forbidden sequences
        \item \blue{Process algebra} to generate models
        \item $\ger{M},2 ~\models~ \red{[a]}\,\text{false}$
      \end{itemize}
    \end{columns}    
  \end{block}
\end{frame}


\begin{slide}{Syllabus (recall)} 
\begin{itemize}
\item Introduction to software architecture
\item \rdb{Background}
\begin{itemize}
\item \dkb{Introduction to transition systems} 
  (\dgold{\textsf{mCRL2}})
\item \dkb{Introduction to modal, hybrid and dynamic logic}
  (\dgold{\textsf{mCRL2}})
\end{itemize}
\item \rdb{Models and calculi of reactive systems}
\begin{itemize}
%\item \dkb{Classical (non deterministic)}  (\dgold{\textsf{mCRL2}})
\item \dkb{Timed (with real time constraints)} (\dgold{\textsf{Uppaal}})
%\item \dkb{Probabilistic}  (\dgold{\textsf{PRISM}})
%\item \dkb{Cyber-physical} (\dgold{\textsf{KeYmaera}})
% \item \transp[30]{Probabilistic  ({\textsf{PRISM}})}
% \item \transp[30]{{Cyber-physical} ({\textsf{KeYmaera}})}
\end{itemize}
\item \rdb{Architecture for reactive systems}
\begin{itemize}
%\item \dkb{An architectural description language} (\dgold{\textsf{AADL}})
\item \dkb{Component-oriented architectural design}
\begin{itemize}
\item  \dkb{Paradigm:} Software components as monadic Mealy machines
%\item \dkb{Foundations:}  Coalgebra theory as a semantic framework for state-based systems a component calculus
\item \dkb{Method:} The \dgold{\textsf{mMm}} calculus; prototyping in  \dgold{\textsc{Haskell}}
\end{itemize}
\item  \dkb{Coordination-oriented architectural design}
\begin{itemize}
\item  \dkb{Paradigm:} The  \dgold{\textsf{Reo}} exogenous coordination model
\item \dkb{Method:} Compositional specification of the glue layer 
\end{itemize}
\end{itemize}
\end{itemize}
\end{slide}


\end{document}
